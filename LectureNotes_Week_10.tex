\documentclass{article}
\usepackage[margin=2cm,bottom=2cm]{geometry}
\usepackage{hyperref}
\usepackage{comment}
\usepackage[utf8]{inputenc}
\usepackage{graphicx}
\usepackage{mathtools}
\usepackage[normalem]{ulem}
\usepackage{color}
\usepackage{setspace}
\usepackage{MnSymbol}
\usepackage[T1]{fontenc}
\usepackage{soul}

\newcommand\tab[1][1cm]{\hspace*{#1}}
\DeclareMathSizes{10}{10}{10}{10}

\begin{document}
\title{COMP/CMPE 314 - Principles of Programming Languages - Notes}
\author{Chris Stephenson, Istanbul Bilgi University, Department of Mathematics, and course students}
\maketitle

\section*{Haskell}
Lots of people stealing ideas from Haskell. Example Java 8 , Scala.

Microsoft Bought Simon Peyton-Jones Glasgow Haskell compiler to who makes their own Haskell -> F# Language

It is invented because lots of different function language used in universities.
\begin{itemize}

 \item 1) Purely Functional -> function values
are first class value
- no assignments no side effects.

 \item 2) Lazy (Example : all list can be infinite. No special treatment is needed for infinite list)

 \item 3) Sophisticated Type System (That allows generic programming)
Variable Types

 \item 4) Lots of interesting sugar
\end{itemize}

I want you to write some programs on Haskell.
\bigskip

\textbf {"Java is no the solution
Java is the problem"} \\


Java doesn't really allow generic programming\\

\section*{Programming Practice}
\begin{verbatim}

"HelloWorld"    

4 + 5             

((+) 4 5)

(+) 4 5     

fac 0 = 1

fac n = fac(n-1) * n

fac 0 1

fac :: Integer -> Integer

[1,2,3] 0:[1,2,3] [0,1,2,3]

GHCI

ones = 1 : []
ones = 1 : ones
CTRL C to stop infinite loop

take 10 ones
take 100 ones

quicksort _ [] = []
quicksort before (x:xs) = [y|y <- xs , before y x]
              ++ [x] 
              ++ [y | y <- xs , not (before y x)]
            
quicksort before (x:xs) = quicksort [y|y <- xs , before y x]
quicksort before (x:xs) = quicksort before [y|y <- xs , not (before y x)]

evens[] = []
evens[x:xs] = x : odds xs
odds[] = []
odds(x:xs) = evens xs

merge :: (a -> a -> Bool) -> [a] -> [a] -> [a]
merge x [] = x
merge [] x = x

merge (x:xs)(y:ys) = ..... 
merge before (x:xs)(y:ys)


mergesort [] = []
mergesort before [x] = [x]
mergesort before (x : xs)(y : ys)
         | before x y = x: merge before xs (y:ys)
         | otherwise = y: merge before (x:xs) ys
  \end{verbatim}

\section*{Beta Transformations}

 (( \lambda m ((\lambda n (((\lambda x (\lambda y (x y)) m) n)))) y) x) \\
 ( \lambda m (\lambda n (((((\lambda x (\lambda y (x y))) m) n))) y) x)\\

Why did we object to identifier Capture?
\bigskip

If we have identifier capture, the result of the B Transform depends on the order they are in .\\

Is it the case that our definition of Beta Transformation (including substitution rule) avoids the problem.\\


The Church-Rosser property a.k.a. \textbf{the Diamond Property}\\


If a λ-sentence M can be transformed by chains of B transformations into sentences H1 and H2 , then there exist (possibly empty) chains of B transformations that transform M and M2 into some sentence M3.
\bigskip

\textbf{A Normal Form} , is a  sentence in which no B transformation is possible
\bigskip

If Church-Russer applies every sentence has at most one normal form.
\bigskip

Google Go -> Says Probably Be
\bigskip

\textbf{Proof }\\
Suppose M has 2 normal forms Church- Russer tells us that the changes of Beta transformations that tansform M1 M2 into M3 but M1 M2 are normal forms so those chains must be empty So M1 = M2 = M3

\bigskip
((\lambda x y)(\lambda x (x x))(\lambda x (x x)))\\
 ----- ---------------------\\
1) y is answer | 2) Goes Infinite
\bigskip

\begin{flushleft}

2 Different beta Transformations one goes to the infinite. According to Diamond Rule, Beta transform should start at the left most space (λx y) to give an answer.\\
\end{flushleft}
Lazy Definition  can find answer if there is a normal form .



\end{document}