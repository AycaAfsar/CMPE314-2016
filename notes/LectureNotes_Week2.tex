\documentclass{article}
\usepackage[margin=2cm,bottom=2cm]{geometry}
\usepackage{hyperref}
\usepackage{comment}
\usepackage[utf8]{inputenc}
\usepackage{graphicx}
\usepackage{mathtools}
\usepackage{color}
\usepackage{amsmath}
\graphicspath{ {bootstrapping.png} }

\newcommand\tab[1][1cm]{\hspace*{#1}}

\begin{document}
\title{COMP/CMPE 314 - Principles of Programming Languages - Notes}
\author{Chris Stephenson, Istanbul Bilgi University, Department of Mathematics, and course students}
\maketitle


\section*{Function/Methods}
\subsubsection*{Roberts on methods:}
  \begin{itemize}
        \item "Hiding complexity"
		\item "Tools for programmers"
		\item "Method calls as expressions"
		\item "Method calls as messages"
   \end{itemize}
 
\begin{flushleft}
The calling mechanism is that the actual parameters are copied to the formal parameters and the code is executed.
\emph{\color{red} it's a lie!}\\

\end{flushleft}
\subsection*{How methods are used}
Within a method you can call another method.(We have certainly done that $\rightarrow$ example Math-min)

\begin{flushleft}
$\Rightarrow {\color{red}Missing:}$
You can call a method in the expression for the actual parameters. $\rightarrow$ myMethod(Math.max(a,b));
\end{flushleft}
\begin{flushleft}
 Used for decomposition - Break a problem into pieces
 \end{flushleft} 
\begin{flushleft}
Used for algorithms - abstraction (example binary search)
\end{flushleft}


\newcommand*{\Comb}[2]{{}^{#1}C_{#2}}%

\textbf{\begin{center}
$\binom nr=\frac{n!}{(n-r)! r!}$
$=\frac{15!}{(15-2)!2!}= 105$
\end{center}}

We wrote \underline{a calculator} to make a programming language, we need to add \underline{abstraction}. (= generalisation. In place of numbers we will use \underline{identifiers} in our calculations.)
\begin{flushleft}
Example: 
\end{flushleft}
\begin{verbatim}
 int square(int x){ 
  return x * x;
 }
\end{verbatim}
$Note:$ x is formal parameter and bound identifier


\begin{flushleft}
\underline{application}
\end{flushleft}
\begin{verbatim}
 println(" " + square(7));
\end{verbatim}
$Note:$ "7" is the actual parameter

\begin{flushleft}
\tab{We evaluate the body of the function by substituting the \underline{actual} parameter for the formal parameter in the body of the function.}\\
\end{flushleft}
\tab{We need to extend our data definition.}\\
\tab{We need to add;}\\
\begin{itemize}
 \item Function definition
 \item Function application
\end{itemize}
\tab{To start with we will keep our function definitions seperate.}\\
\tab{Our functions will have a seperate \underline{namespace}.}\\
\tab{Our interpreter will have two inputs - a list of function definitions - an expression.}\\
(Which may include function applications)
  
\subsection*{What does a function have?}
\begin{itemize}
 \item A formal parameter - name (identifier)
 \item A body - expression (in our extended version)
 \item A name - identifier (in the function name namespace)
 \item Function name - identifier in the function name namespace
 \item Actual parameter - identifier
\end{itemize}
\tab{Extend our expression data definition.}\\
\tab{We need to add two more variants.}\\
\begin{itemize}
 \item Function application
 \item Identifier
\end{itemize}
\tab{We need to extend the interpreter to deal with the two new variants in the data definition.}
\begin{itemize}
 \item Identifier $\rightarrow$ error
 \item Function application\\
 \verb|(apply (find-function function-list function-name) actual-parameter)|
\end{itemize}
\tab{Statement of purpose for apply, substitute the actual parameter for the formal parameter everywhere in the body of the function, then evaluate the result.}

\begin{verbatim}
 (define '(apply function-definition actual-parameter){
  (interp (subst (function-definition formal-parameter) actual-parameters))
 })
\end{verbatim}



\begin{flushleft}
\textbf{What does subst do?}\\
\end{flushleft}
name expression expression $\rightarrow$ expression

\begin{flushleft}
\underline{Template:}\\
\end{flushleft}
\begin{flushleft}
number $\rightarrow$ number\\
arith-expression $\rightarrow$ same expression but subst the operands\\
identifier $\rightarrow$ if the identifier is the identifier to be substitued, we replace it\\
function application $\rightarrow$ function application, but subst the actual parameter\\
\end{flushleft}
\textbf{Valid sentences:}\\
$\mathit{a}$\\
($\mathit{a}$ $\mathit{b}$)\\
($\lambda$ $\mathit{x}$ $\mathit{x}$) - informally this is the identity function\\
{($\mathit{a}$ $\mathit{b}$ $\mathit{c}$)} - {not valid}

($\lambda$ $\mathit{f}$ ($\lambda$ $\mathit{x}$ ($\mathit{f}$ 
($\mathit{f}$ $\mathit{x}$)))) - a function doubler\\


\end{document}