\documentclass{article}
\usepackage[margin=2cm,bottom=2cm]{geometry}
\usepackage{hyperref}
\usepackage{comment}
\usepackage[utf8]{inputenc}
\usepackage{graphicx}
\usepackage{mathtools}
\usepackage[normalem]{ulem}
\usepackage{setspace}
\usepackage{MnSymbol}
\usepackage{soul}

\DeclareMathSizes{10}{10}{10}{10}

\begin{document}
\title{COMP/CMPE 314 - Principles of Programming Languages - Notes}
\author{Chris Stephenson, Istanbul Bilgi University, Department of Mathematics, and course students}
\maketitle

\section*{Laziness in Racket}
\begin{flushleft}
\underline{CONS} - We want it to be lazy\\
How can we get laziness in an eager language?\\
\begin{verbatim}
 (lambda () <expr>) --> promise
\end{verbatim}
to force our promise to evaluate we just write \verb|(promise)|\\
\begin{verbatim}
 (delay (+ 2 3))
\end{verbatim}
\verb|define-syntax-rule| makes a function that operates on program text. It's a way of providing syntactic sugar. 
\end{flushleft}

\section*{Hamming Code}
\begin{flushleft}
 \underline{Hamming's Problem}\\
 Produce a list of numbers only divisible by the prime numbers 3, 5, 7 and no other prime number in order 1 3 5 7 9 15 21 25 27 35
\end{flushleft}

\section*{Georg Contor}
\begin{flushleft}
The hotel with an infinite number of rooms is full.\\
\bigskip
\begin{tabular}{c | c c c c}
  - & 1 & 2 & 3 & 4 \\
  \hline
  0 & / & / & / & / \\
  1 & / & / & / &  \\
  2 & / & / &  &  \\
  3 & / &  &  &  \\
\end{tabular}
\bigskip
\\What is our interpreter now going to look like?\\
We have \underline{function values}. So we need a different way of naming functions.\\
The only way we now have of creating names is as function parameters.\\
This is cumbersome to write in the target language.\\
Why don't we use the syntactic sugar of a Racket style let statement?
\pagebreak
\begin{verbatim}
 (let
    <def. list>
    expr)
    
 <def. list> --> ((name1 expr1)
                  (name2 expr2)
                  (namen exprn)
                  ...)
                  
 (let <def. list> expr) --> (let ((name1 expr1))
                               (let ((name2 expr2))
                                  (let ...
                                     expr)))
\end{verbatim}
desugar2

\verb|(let (name exp) program)| $ \xrightarrow{desugar} $ \verb|((lambda name program) expr)| \\

\subsection*{Grammar for our target language:}
\begin{itemize}
 \item[] S $\rightarrow$ identifier
 \item[] S $\rightarrow$ (S S)
 \item[] S $\rightarrow$ ($\lambda$ identifier S)
 \item[] S $\rightarrow$ (let (identifier S) S)
 \item[] S $\rightarrow$ number
 \item[] S $\rightarrow$ (+ S S) ...
\end{itemize}
When we want multi-parameter functions we can also do this by desugaring.\\
\bigskip
\verb|(lambda (| $v_1$ $v_2$ $v_3$ \verb|expr)| $\xrightarrow{desugar}$ \verb|(lambda| $v_1$ \verb|(lambda| $v_2$ \verb|(lambda| $v_3$ ... \verb|expr))))|\\
\bigskip
\verb|(a b c d)| $\xrightarrow{desugar}$ \verb|(((a b) c) d)|
\end{flushleft}

\subsection*{$\lambda$-calculus}
\begin{flushleft}
  \begin{itemize}
   \item[] (($\mathit{I}$ $\mathit{M}$) $\mathit{N}$) $\rightarrow$ ($\mathit{M}$ $\mathit{N}$)
   \item[] exponentiation = identity function
  \end{itemize}
The predecessor function is \underline{hard.}\\
\verb|(PRED n)| what we want it \verb|n-1|\\
Think of a pipeline\\
\bigskip
Let us program this transformation on pairs.\\
$\mathit{[} \mathit{a,b} \mathit{]}$  $\rightarrow$ $\mathit{[} \mathit{b, (SUCC}$ B$) \mathit{]}$

\begin{verbatim}
 (lambda n
    (FIRST ((n (lambda c
                  ((CONS (REST c)) (SUCC (REST c))))))
                  ((CONS ZERO) ZERO)))
\end{verbatim}
This is a \verb|PRED| function.\\
\pagebreak
\subsection*{Observation}
\begin{tabular}{l c l}
 ($\lambda$ x x)		& $\equiv$ & ($\lambda$ y y) ($\alpha$ transformation)\\
 ($\lambda$ x ($\lambda$ y y))	& $\equiv$ & ($\lambda$ a ($\lambda$ b b)) ($\alpha$ transformation)\\
 ($\lambda$ x ($\lambda$ y x))	& $\neq$   & ($\lambda$ a ($\lambda$ b b))\\
\end{tabular}
\bigskip
\\What is the normal form?\\
What is the essential difference?\\
\end{flushleft}
\bigskip
\begin{flushleft}
\subsection*{de Bruijn indices}
\begin{tabular}{l c l}
  ($\lambda$ x x)		& $\rightarrow$		& ($\lambda$ 1)\\
  ($\lambda$ ($\lambda$ 1))	& $\rightarrow$		& ($\lambda$ ($\lambda$ 1))\\
  ($\lambda$ ($\lambda$ 2))	& $\rightarrow$		& ($\lambda$ ($\lambda$ 1))\\
\end{tabular}

\end{flushleft}

\end{document}
