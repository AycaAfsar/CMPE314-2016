\documentclass{article}
\usepackage[margin=2cm,bottom=2cm]{geometry}
\usepackage{hyperref}
\usepackage{comment}
\usepackage[utf8]{inputenc}
\usepackage{mathtools}
\usepackage{color}
\usepackage{amsmath}
\usepackage{array}
\usepackage{setspace}
\usepackage{soul}
\usepackage{amssymb}
\usepackage{graphicx}

\begin{document}
\title{COMP/CMPE 314 - Principles of Programming Languages - Notes}
\author{Chris Stephenson, Istanbul Bilgi University, Department of Mathematics, and course students}
\date{April,7}
\maketitle

\section{Haskell}
\begin{flushleft}
\begin{itemize}
\item Purely functional 
\item Statically typed 
\item Lazy.
\end{itemize}
\end{flushleft}

\subsection*{Haskell Curry}
\begin{flushleft}
\begin{itemize}
\item Haskell Platform
\item Simon Reyton-Jones (F\#) (Glasgow) 
\end{itemize}
\end{flushleft}

\section{Functional Programming}
\begin{flushleft}
\begin{itemize}
\item Functions are first class values
\item No state / No side effects\\
\doublespacing
How can we do this in our interpreter ?\\
A function application (name expr) will no longer be, it will be (expr expr);\\
$\downarrow$
\item This means we have more than one \underline{type} of expression
\item So expression values are no longer automatically numbers.
\item So we have runtime types.
\end{itemize}
\end{flushleft}

\subsection*{1- TYPES}
\subsection*{2- How will functions be named ?}
\begin{flushleft}
\begin{itemize}
\item Just like any other expression.
\item Identifiers (which are always formal parameters) and are then applied to actual parameters.
\item This will also be true for functions.
\end{itemize}
\doublespacing
What would be like to have?

(let
   
  \hspace{0.5cm}(myfunc ($\lambda$ x ( + x 3 ))) $\rightarrow$ function def.\\ 
  \hspace{0.5cm}(myfunc2 ($\lambda$ y (+ y 4 ))) $\rightarrow$ function def.\\ 
  \hspace{0.5cm}\textless\ program \textgreater {} $expression$
\\)


\textbf{After desugaring:}

((  $\lambda$ (myfunc myfunc2)\\
\textless\ program \textgreater)\\
\hspace{0.5cm}( $\lambda$ (x) (+ x 3))\\
\hspace{0.5cm}( $\lambda$ (y) (+ y 4)))\\
\textbf{Let is sugar.}\\

(let
   
  \hspace{0.5cm}((double ($\lambda$ (x) ( * x 2 )))\\
  \hspace{0.5cm}(double 7))\\
   
(($\lambda$ (double)\\
\hspace{0.5cm}(double 7))\\
\hspace{0.5cm}($\lambda$ (x) ( * x 2 )))\\
\doublespacing

(let
   
  \hspace{0.5cm}((multiply-list-by-n ($\lambda$ (l n) (map \underline{($\lambda$ (x) (* x n))} l)) // Some function def. Does something different\\
  \hspace{1cm}(multiply-list-by-n '(4 5) 3)\\
  \hspace{1cm}(multiply-list-by-n '(5 6) 7)\\

\end{flushleft}


\subsection*{3- Closure}
\begin{flushleft}
We need to distinguish function definitions from \underline{function values} when we evaluate a function definition, we get a \underline{function value}. \\

\doublespacing
function definition $(one$ $\rightarrow$ $many)$ function values.\\

A function value is a function definition plus the environment at the point where the function value was evaluated. This is called a \textbf {closure.}
\end{flushleft}

\subsection*{4- When we apply a function value?}

Extend the environment in the function value with the bindings of the formal parameters to the actual parameters
\doublespacing
\textbf{Sugaring the $\lambda$-calculus; }\\
\vspace{2mm} Some definitions;

($F^{0}$M)  $\equiv$ M\\
($F^{n+1}$M)  $\equiv$ F(($F^{n}$M)\\

Church Numeral = $C_n$\\
$C_n$ $\equiv$($\lambda$ f ($\lambda$ x ( $f^{n}$ x )))\\
$C_0$ $\equiv$($\lambda$ f ($\lambda$ x ( $f^{0}$ x )))  $\equiv$ ( $\lambda$ f ( $\lambda$ x x )) \\
$C_1$ $\equiv$($\lambda$ f ($\lambda$ x x ( $f^{1}$ x )))  $\equiv$ ( $\lambda$ f ( $\lambda$ x (f x ))) \\
$C_2$ $\equiv$($\lambda$ f ($\lambda$ x x ( $f^{2}$ x )))  $\equiv$( $\lambda$ f ( $\lambda$ x f (f x ))) \\
\doublespacing\\
\underline{LET} - as we did for our interpreter.\\
\textbf{SUCC} ($\lambda$ n ( $\lambda$ f ( $\lambda$ x) (f (( n f) x ))))\\
\textbf{TRUE} ( $\lambda$ x ( $\lambda$ y x ))\\
\textbf{FALSE} ( $\lambda$ x ( $\lambda$ y y ))\\
\textbf{IF} ( $\lambda$ a ( $\lambda$ b ($\lambda$ c ((a b ) c))))\\
\textbf{ZERO?} ( $\lambda$ n (( n ($\lambda$ x FALSE)) TRUE))\\


\section{Data Structure}

If we can make a pair, we can make a data structure.\\
What are the semantics of CONS FIRST REST (in Racket)?\\
\doublespacing\\
\hspace{3mm} (FIRST ( CONS A B )) $\equiv$ A // Necessary semantics\\
\hspace{3mm} (REST (CONS A B ))  $\equiv$ B // Necessary semantics\\
\doublespacing\\
\textbf{CONS:} ($\lambda$ a ($\lambda$ b ( $\lambda$ \textcolor{red}{((f a ) b)})))\\
\textbf{FIRST:} ($\lambda$ c (c TRUE))\\
\textbf{REST:} ($\lambda$ c (c FALSE))\\


\end{document}