\documentclass{article}
\usepackage[margin=2cm,bottom=2cm]{geometry}
\usepackage{hyperref}
\usepackage{comment}
\usepackage[utf8]{inputenc}
\usepackage{graphicx}
\usepackage{mathtools}
\usepackage[normalem]{ulem}
\usepackage{setspace}
\usepackage{MnSymbol}
\usepackage{soul}

\newcommand\tab[1][1cm]{\hspace*{#1}}
\DeclareMathSizes{10}{10}{10}{10}

\begin{document}
\title{COMP/CMPE 314 - Principles of Programming Languages - Notes}
\author{Chris Stephenson, Istanbul Bilgi University, Department of Mathematics, and course students}
\maketitle

\section*{Haskell}
\begin{itemize}
 \item Purely functional - no state of programs
 \item Statically typed
 \item Lazy
\end{itemize}

\section*{Functional Programming}
\begin{itemize}
 \item Functions are first class values.
 \item No state / no side effects
\end{itemize}
\begin{flushleft}
 How can we do this in our interpreter?\\
 A function application \verb|(name expr)| will no longer be.\\
 It will be \verb|(expr expr)|\\
 This means we have more than one \underline{type} of expression. So runtime types. So we can have booleans.
\end{flushleft}
\section{Types}
\section{How will functions be named?}
\begin{flushleft}
\begin{itemize}
 \item Just like any other expression.
 \item Identifiers (which are always formal parameters) and are then applied to actual parameters.
 \item This will also be true for functions.
\end{itemize}
\bigskip
What would we like to have?\\
\begin{verbatim}
 (let
    (myfunc (lambda (x) (+ x 3)))      function definition
    (myfunc2 (lambda (y) (* y 4)))     function definition
    ...<program>                       expressions
    )
\end{verbatim}
\underline{let} is \underline{sugar}.\\
Desugaring the let;
\begin{verbatim}
 ((lambda (myfunc myfunc2)
    <program>)
    (lambda (x) (+ x 3)) (lambda (y) (* y 4)))
\end{verbatim}
\pagebreak
\underline{Example:}
\begin{verbatim}
 (let
      ((double (lambda (x) (* x 2)))
    (double 7))
 
 ((lambda (double)
    (double 7))
      (lambda (x) (* x 2)))
\end{verbatim}
\begin{verbatim}
 (let
    ((multiply-list-by-n (lambda (l n) (map (lambda ((x) (* x n)) l))
      (multiply-list-by-n '(4 5) 3)
      (multiply-list-by-n '(5 6) 7)))))
\end{verbatim}
\end{flushleft}

\section{Closure}
\begin{flushleft}
 We need to distinguish function definitions from \underline{function values} when we \underline{evaluate} a function definition we get a \underline{function value}.\\
 \bigskip
 function definition $\xrightarrow{one->many}$ function value\\
 \bigskip
 A \underline{function value} is a function definition \underline{plus} the environment at the point where the function value was evaluated. This is called \underline{closure}.
\end{flushleft}

\section{When we apply a function value?}
\begin{flushleft}
 Extend the environment in the function value with the bindings of the formal parameters to the actual parameters.
\end{flushleft}
\subsection{Sugaring the $\lambda$-calculus}
\begin{flushleft}
 \underline{Some definitions}\\
 ($F^{0}$M)  $\equiv$ M\\
 ($F^{n+1}$M)  $\equiv$ F($F^{n}$M))\\
 \bigskip
 $C_n$ = Church numeral $\mathit{n}$\\
 \bigskip
 So,\\
 $C_n$ $\equiv$($\lambda$ $\mathit{f}$ ($\lambda$  $\mathit{x}$  ( $f^{n}$  $\mathit{x}$)))\\
 $C_0$ $\equiv$($\lambda$ $\mathit{f}$ ($\lambda$  $\mathit{x}$  ( $f^{0}$  $\mathit{x}$)))  $\equiv$ ($\lambda$ $\mathit{f}$ ($\lambda$  $\mathit{x}$   $\mathit{x}$))\\
 $C_1$ $\equiv$($\lambda$ $\mathit{f}$ ($\lambda$  $\mathit{x}$  ($f^{1}$  $\mathit{x}$)))  $\equiv$ ($\lambda$ $\mathit{f}$ ($\lambda$  $\mathit{x}$  ($\mathit{f}$  $\mathit{x}$)))\\
 $C_2$ $\equiv$($\lambda$ $\mathit{f}$ ($\lambda$  $\mathit{x}$  ($f^{2}$  $\mathit{x}$)))  $\equiv$ ($\lambda$ $\mathit{f}$ ($\lambda$  $\mathit{x}$  ($\mathit{f}$ ($\mathit{f}$  $\mathit{x}$))))\\
 \bigskip
 \underline{LET} - as we did for our interpreter\\
 \begin{tabular}{l l}
  SUCC & ($\lambda$ $\mathit{n}$ ($\lambda$ $\mathit{f}$ ( $\lambda$ $\mathit{x}$) ($\mathit{f}$ (($\mathit{n}$ $\mathit{f}$) $\mathit{x}$))))\\
  TRUE & ($\lambda$ $\mathit{x}$ ($\lambda$ $\mathit{y}$ $\mathit{x}$))\\
  FALSE & ($\lambda$ $\mathit{x}$ ($\lambda$ $\mathit{y}$ $\mathit{y}$))\\
  IF & ($\lambda$ $\mathit{a}$ ($\lambda$ $\mathit{b}$ ($\lambda$ $\mathit{c}$ (($\mathit{a}$ $\mathit{b}$) $\mathit{c}$))))\\
  ZERO? & ($\lambda$ $\mathit{n}$ (($\mathit{n}$ ($\lambda$ $\mathit{x}$ $\mathit{FALSE}$)) $\mathit{TRUE}$))\\
 \end{tabular}
 \end{flushleft}
\pagebreak
 \subsection{Data Structures}
 If we can make a pair, we can make a data structure.\\
 What are the semantics of \verb|CONS FIRST REST| (in Racket)?\\
 \begin{itemize}
  \item[] \verb|(FIRST (CONS A B)) = A        -> necessary semantics|
  \item[] \verb|(REST (CONS A B)) = B         -> necessary semantics|
 \end{itemize}
 \bigskip
\begin{tabular}{l l}
 CONS & ($\lambda$ $\mathit{a}$ ($\lambda$ $\mathit{b}$ ($\lambda$ $\mathit{f}$ (($\mathit{f}$ $\mathit{a}$) $\mathit{b}$))))\\
 FIRST & ($\lambda$ $\mathit{c}$ ($\mathit{c}$ $\mathit{TRUE}$))\\
 REST & ($\lambda$ $\mathit{c}$ ($\mathit{c}$ $\mathit{FALSE}$))\\
\end{tabular}
\end{document}
