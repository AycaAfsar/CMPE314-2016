\documentclass{article}
\usepackage[utf8]{inputenc}
\usepackage{hyperref}
\usepackage{comment}
\usepackage[utf8]{inputenc}
\usepackage[margin=2cm,bottom=2cm]{geometry}

\title{CMPE314 Notes Week1}
\date{February 2016}


\begin{document}

\maketitle

\section{Lecture 01}
\subsubsection*{Quiz Question 1 answer:}
    
\begin{itemize}

\item  Language: 
    \begin{itemize}
        \item A finite alphabet.
        \item A set of strings of the symbols in the alphabet. (Usually infinite)

    \end{itemize}

\item Grammar:
    \begin{itemize}
        \item A finite alphabet of terminal symbols.
        \item A finite set of non-terminal symbols.
        \item  A set of productions (rewriting rules).
        \item Contains a sentence symbol S. (start)
        \item Simple Grammar example: 
    
        Alphabet: [a]
    
        S  $\rightarrow$  a
    
        Language: Set consisted of a.
        
        \item Quiz 01 grammar corresponding to the Quiz Question 1
        
        Alphabet: [+, *, (), number]
        
        S  $\rightarrow$ number
        
        S  $\rightarrow$ S+S 
        
        S  $\rightarrow$ S*S
        
        S $\rightarrow$ (S)
        
        Push down automaton can recognize this language.
        
        
        \end{itemize}
        
\end{itemize}

\subsubsection*{Quiz Question 2 answer:}

\begin{verbatim}
    (define (interp [a : ArithC]) : number
        (type-case ArithC a
            [numC (n) n]
            [plusC (l r) (+ (interp l) (interp r))]
            [multC (l r) (* (interp l) (interp r))]))

\end{verbatim}
Reason that \textbf{numC-n} which was previously written instead of \textbf{interp} is that \textbf{l} or
\textbf{r} corresponding to the \textbf{plusC} or \textbf{mulC} do not have necessarily to be of the \textbf{numC}
type, instead they can be another \textbf{plusC} or  \textbf{mulC} themselves meaning that \textbf{interp} should
be recursively called for both \textbf{l} and \textbf{r}.

Ex: (+ (* 3 7) (+ 4 2))

\begin{flushleft}
What programs mean is not a trivial business.

The code below is valid C and also valid Java.

What value does the function/method \verb|haha| return?
%\end{flushleft}


\begin{verbatim}
 int funny (int a, int b) {
     return a + 2 * b;
 }
 
 int haha() {
     int c = 2;
     return funny (c++, c++);
 }
\end{verbatim}

What would the value of these expressions be in Java and in Python and in PHP? Or any other language?
=======
Returned value in Java is 8,while in C it is 7.

What would the value of these expressions be in Java and in Python and in PHP?
\begin{verbatim}
 1000000000 + 2000000000
 "1000000000" + "2000000000"
\end{verbatim}

Since java takes the first line as integers the sum of these two numbers would be too big for what Java's int can take, therefore the answer would be some negative number.

Racket for example can do this operation and would return the result as 3000000000.


The second line Java sees as a string therefore the result would be an concatenated string "10000000002000000000".

On the other hand some languages would calculate the second line as 3000000000.
\setlength{\parindent}{10ex}

\par Interesting fast about languages is that they do have Syntactic Sugar (decoration of language). An simple example of it is a dowhile loop in Java, there is nothing new that it brings to Java, nothing that while or for loop can no do. Taking this out of Java syntax would be called \textbf{desugaring}.

\noindent
Similar is with OR AND and NOT gates in circuits. For ex.: while NAND (AND + NOT) is an complete sent and with combination of few And gates the function of the XOR can be performed. Meaning that in this case XOR gate is a syntactic sugar.

All programming languages are data formats for data input to other programs.

\noindent
Ex: \textbf{Java code} is input for \textbf{compiler} (which is program itself) and it produces a ByteCode (which is program again) and it is input for JVM (which is program itself) and so on...

\noindent
If this is the case, where does this process end ? 

\noindent
The Manchester Small-Scale Experimental Machine (SSEM), nicknamed Baby, was the world's first stored-program computer. It was built at the Victoria University of Manchester, England, by Frederic C. Williams, Tom Kilburn and Geoff Tootill, and ran its first program on 21 June 1948.

\noindent
At that time for opening the computer small binary code was manually, nowadays it is not needed since there is ROM memory. Processor is fixed to jump to a fixed memory address, that is how it starts. 
                
Program Text  $\rightarrow$ $\rightarrow$ Data Structure(CANONICAL FORM)  $\rightarrow$ $\rightarrow$- Answer 

Program Text is \textbf{parsed} into a Data structure, and Data Structure is \textbf{interpreted} into answer. The focus of this course will be more on the second part of the process interpreting Data Structure.

\end{flushleft}


\section*{Primary Textbooks}
\begin{itemize}


\item Shriram Krishnamurthi \emph{Programming Languages: Application and Interpretation} Author's web site  second edition. 

\url{http://cs.brown.edu/courses/cs173/2012/book/} (primary text)

\item Daniel P. Friedman and Mitchell Wand, \emph{Essentials of Programming Languages}, 3rd Edition MIT Press 2008
\end{itemize}


\subsection*{Background reading}
\begin{itemize}

\item Harold Abelson, Gerald Jay Sussman, Julie Sussman 
\emph{Structure and Interpretation of Computer Programs}
Second edition, MIT Press 2001

Available on-line at 

\url{http://mitpress.mit.edu/sicp/full-text/book/book.html}

\item Scott Chacon and Ben Straub \emph{Pro Git}
Available on-line at 

\url{http://git-scm.com/book/en/v2}
\item Barendregdt \emph{A Short Guide to the $\lambda$-calculus}.  
I have uploaded a copy to online.bilgi.edu.tr

\item \emph{Why Undergraduates Should Learn the Principles of
Programming Languages} online at

\url{http://www.cs.pomona.edu/~kim/why.pdf}

\end{itemize}


\section*{Video}
A complete set of video lectures for a previous version of this course is available at

\url{https://vimeo.com/album/1987162}


\noindent{\copyright{Chris Stephenson 2016}}

\end{document}
