\documentclass{article}
\usepackage[margin=2cm,bottom=2cm]{geometry}
\usepackage{hyperref}
\usepackage{comment}
\usepackage[utf8]{inputenc}
\usepackage{mathtools}
\usepackage{color}
\usepackage{amsmath}
\usepackage{array}
\usepackage{setspace}
\usepackage{soul}
\usepackage{amssymb}
\usepackage{graphicx}

\begin{document}
\title{COMP/CMPE 314 - Principles of Programming Languages - Notes}
\author{Chris Stephenson, Istanbul Bilgi University, Department of Mathematics, and course students}
\date{April,14}
\maketitle

\section{Laziness}
\begin{flushleft}
\textcolor{red}{\textbf{\underline{Cons:}}} We want it to be lazy. How can we get laziness in an eager language?\\

\doublespacing
($\lambda$ () \underline{$<$expr$>$}) $\rightarrow$ promise\\
\doublespacing
to force our promise to evaluate we just unite (promise)\\
\doublespacing
(delay \underline{(+ 2 3)})\\
\doublespacing
define-syntax-rule makes a function that operates on program text. It is a way of providing syntactic sugar.
\end{flushleft}


\subsection*{Hamming Code}
\begin{flushleft}
Reduce a list of numbers only divisible by the prime numbers 3,5,7 and no other prime number in order 1  3  5  7  9  15  21  25  27  35
\end{flushleft}

\subsection*{Georg Contor}
\begin{flushleft}
The hotel with a infinite number of room is full\\

\doublespacing
What is our interpreter now going to look like?

\doublespacing
We have \underline{function values}. So we need a different way of naming functions

\doublespacing
The only way we now have of creating names is as function parameters

\doublespacing
This is cumbersome to write in the target language.

\doublespacing
Why don't we use the syntactic sugar of a Racket style Let statement?
\begin{verbatim}
(let
  <def list>
    exp)
    ...
   <def list> ---> ((name1 expr1)
                    (name2 expr2)
                    (namen exprn)..)

(let <def list> exp) ---> (let ((name1 expr1))
                              (let ((name2 expr2)) 
                                 (let... expr))))
\end{verbatim}
desugar 2\\
(let (name exp) program) $\xrightarrow{desugar}$ (($\lambda$.name program) expr)
\end{flushleft}

\subsection*{Grammer for our target language}
\begin{flushleft}
S $\rightarrow$ identifier\\
S $\rightarrow$ (S S)\\
S $\rightarrow$ ($\lambda$ identifier S)\\
S $\rightarrow$ (let (identifier S) S)\\
S $\rightarrow$ number\\
S $\rightarrow$ (+ S S)\\

\doublespacing
When we want to multi-parameter functions. We can also do this by desugaring. \\

\doublespacing
($\lambda$ (V1 V2 V3) expr) $\xrightarrow{desugar}$ ($\lambda$ V1 ($\lambda$ V2 ($\lambda$ V3 ... expr))))\\

\doublespacing
(a b c d) $\xrightarrow{desugar}$ (((a b) c) d)
\end{flushleft}


\subsection*{Lambda Calculus}
\begin{flushleft}
((I M) N) $\rightarrow$ (M N)\\
\doublespacing
exponentation = identity function

\doublespacing
The predecessor function is \underline{hard}\\
(PRED n) what we want is n-1\\
Think of a pipeline\\
Let us this transformation an pairs\\
\center
[a b] $\rightarrow$ [b, (Succ B)]
\begin{verbatim}
(lambdan
    (FIRST ((n (lambda c
        ((CONS (REST c)) (SUCC (REST c))))))
        ((CONS ZERO) ZERO))))
        This is a PRED function
\end{verbatim}
\end{flushleft}

\subsection*{Observation}
\begin{flushleft}
($\lambda$ x x) $\equiv$ ($\lambda$ y y) ($\alpha$ transformation)\\
($\lambda$ x ($\lambda$ y y)) $\equiv$ ($\lambda$ a ($\lambda$ b b)) ($\alpha$ transformation)\\
($\lambda$ \textcolor{red}{x} ($\lambda$ y \textcolor{red}{x})) $\neq$ ($\lambda$ a ($\lambda$ \textcolor{red}{b} \textcolor{red}{b}))\\
 \doublespacing
What is the normal form?\\
($\lambda$ \textcolor{red}{x} ($\lambda$ y \textcolor{red}{x})) $\neq$ ($\lambda$ a ($\lambda$ \textcolor{red}{b} \textcolor{red}{b}))\\
What is the essention difference
\end{flushleft}

\subsection*{de Burjin indices}
\begin{flushleft}
($\lambda$ x x) $\rightarrow$ ($\lambda$ 1)\\
($\lambda$ ($\lambda$ 1)) $\equiv$ ($\lambda$ ($\lambda$ 1))\\
($\lambda$ ($\lambda$ 2)) $\neq$ ($\lambda$ ($\lambda$ 2))
\end{flushleft}
\end{document}