\documentclass{article}
\usepackage[margin=2cm,bottom=2cm]{geometry}
\usepackage{hyperref}
\usepackage{comment}
\usepackage[utf8]{inputenc}
\usepackage{mathtools}
\usepackage{color}
\usepackage{amsmath}
\usepackage{array}
\usepackage{setspace}
\usepackage{soul}
\usepackage{amssymb}
\usepackage{graphicx}

\begin{document}
\title{COMP/CMPE 314 - Principles of Programming Languages - Notes}
\author{Chris Stephenson, Istanbul Bilgi University, Department of Mathematics, and course students}
\date{March,24}
\maketitle

\section*{Evaluation "Strategies"}
\begin{flushleft}
\underline{All about function application}- which is the central concepts in programming \underline{(IMHO)} an \underline{oxymoron.} 
\end{flushleft}

\doublespacing
\begin{flushleft}
- We looked at sequence (2 weeks ago)\\
- Strict vs non-strict\\
- Eager vs Lazy\\
- Left to Right is Right to Left\\
\end{flushleft}

\doublespacing
\begin{flushleft}
What is the passed when we apply a function to actual parameters ?
\end{flushleft}

\doublespacing
\begin{flushleft}
\underline{When?}\\
Example In Java: 
\begin{verbatim}
{ 
a=2;
myFunc(a);
}
------------------------------
void myFunc(int a){
...
a=a+1;
}
\end{verbatim}
\end{flushleft}

\subsection*{What is the value of a?}
\begin{flushleft}
System.out.println("a is" +a); $\leftarrow$ The answer is \underline{2}.
\end{flushleft}

\doublespacing
\begin{flushleft}
\underline{Why?}\\
Because Java / C / C\# / Ruby / Python / C++ are \underline{call by value}\\
For simple types, most common programming languages are call by value.\\
The alternative is \underline{call by preference} (PHP can do this)
\end{flushleft}

\subsection*{But What about complex types ?}
\begin{flushleft}
Examples-array, structure, object.
\begin{verbatim}
{ 
a[ ]=2;
myFunc(a);
System.out.println("a[ ] is" +a); }

void myFunc(int [] a){
---
a[ ]=a[ ]+1
}
\end{verbatim}
\end{flushleft}

\doublespacing
\begin{flushleft}
Call by reference for arrays / structures / Strings / objects 
\begin{verbatim}
[Except C-C structures are naturally call by value]
\end{verbatim}
\begin{flushleft}
Java strings can be considered call-by-value (because Java strings are immutable)
\end{flushleft}
\end{flushleft}

\subsection*{Super accelerated CMPE 313 - (SICP-The Purple Book 1989)}
\begin{flushleft}
What is a List ? A list of numbers ?
\end{flushleft}
\begin{itemize}
\item Empty
\item A pair of a \st{number}(x) and a list of \st{number}(x)
\end{itemize}
\begin{verbatim}
;;add up the numbers in a list
(define (sum-lon lon)
  (cond)
  ((empty? lon) O(empty set))
  (else ..(+ (first lon)...(sum-lon(rest lon)))))
  
(define (mul-lon lon)
  (cond 
  ((empty? lon) 1)
  (else..(* (first lon)(mul-lon(rest lon)))))
  
(define (fold combiner null-value lox)
  (cond
  ((empty? lox) 1)
  (else (* (combiner (first lox) (fold combiner null-value (rest lox))))))
\end{verbatim}

\begin{flushleft}
fold is a powerful \underline{abstraction}.\\
I want to evaluate a \underline{polynomial}.
\end{flushleft}

\begin{center}
${a}_{0} + {a}_{1}x + {a}_{1}{x}^{2} + {a}_{1}{x}^{3} +....+ {a}_{n}{x}^{n}$ 
\end{center}

\begin{flushleft}
Data definition a polynomial is a list of coefficient $({a}_{0},{a}_{1},..,{a}_{n})$\\
Contract eval polynomial number $\rightarrow$ number\\
Horner's rule ${a}_{0} + x({a}_{1} + x({a}_{2} + x({a}_{3}...{a}_{n}))))$\\
To evaluate a polynomial\\
------\\
(define (eval-poly p x)\\
(fold (lambda (l $\gamma$) (+ l (* x r)) $\emptyset$ p))
\end{flushleft}

\begin{flushleft}
\underline{map}\\
(define (map f lox)\\
(fold (lambda (l $\gamma$) (cons (f l) r)) empty lox))))\\
\end{flushleft}

\begin{flushleft}
\underline{filter}\\
(define (filter p? lox)\\
(fold (lambda (L r) (if (p? L) (cons (r) r) empty lox))\\
\end{flushleft}

\doublespacing
\begin{flushleft}
Suppose I am doing a \underline{map} to add 2 to every item in a list
\begin{verbatim}
(define (add2 lon)
  (map (lambda (x) (+ x 2)) lon))
  
(define (mul 7 lon)
  (map (lambda (x) (* x 7)) lon ))
  
(define (carry f x)
  (lambda (y) (f x y)))
---------------------------
(define (arith-map openbar constant)
  (map (carry operator constant) lon))
\end{verbatim}
\end{flushleft}

\begin{flushleft}
\textbf{The $\lambda$-calculus} (Summary)\\
$1)$ $\Lambda\rightarrow\upsilon$

$2)$ $\Lambda\rightarrow(\lambda$ 
$\upsilon$
$\Lambda)$

$3)$ $\Lambda\rightarrow(\Lambda$
$\Lambda)$
\end{flushleft}

\doublespacing
\begin{flushleft}
$\alpha$ ($\lambda$ $\upsilon$ $\Lambda$) $\equiv$ ($\lambda$ $\upsilon$ $\Lambda$ [$\upsilon$:=u])\\
$\gamma$ ($\lambda$ $\upsilon$ (f $\upsilon$)) $\equiv$ f\\
$\beta$ (($\lambda$ $\upsilon$ ${\Lambda}_{1}$) ${\Lambda}_{2}$ $\rightarrow$ ${\Lambda}_{1}$ [$\upsilon$:=${\Lambda}_{2}$]
\end{flushleft}

\begin{flushleft}
Defining substitution is more complicated.\\
$\Lambda$[$\upsilon$:=${\Lambda}_{0}$] - see the slides 
\end{flushleft}

\begin{flushleft}
\textbf{a)} $\Lambda$=u\\
if u=$\upsilon$ $\rightarrow$ ${\Lambda}_{0}$ \\
else $\Lambda$\\
\textbf{b)} $\Lambda$=(${\Lambda}_{1}$ ${\Lambda}_{2}$)\\
(${\Lambda}_{1}$[$\upsilon$=${\Lambda}_{0}$] ${\Lambda}_{2}$[$\upsilon$=${\Lambda}_{0}$]\\
\textbf{c)} $\Lambda$=($\lambda$ u ${\Lambda}_{1}$)\\
(i) u=$\upsilon$ $\rightarrow$ $\Lambda$\\
(ii) u$\neq\upsilon$ if necessary re-letter before substituting in the body.
\end{flushleft}

\subsubsection*{Church Numerals}
\begin{flushleft}
0 $\equiv$ ($\lambda$ f ($\lambda$ x x))\\
1 $\equiv$ ($\lambda$ f ($\lambda$ x (f x)))\\
2 $\equiv$ ($\lambda$ f ($\lambda$ x (f (f x))))
\end{flushleft}

\subsubsection*{Successor}
\begin{flushleft}
Succ=($\lambda$ n ($\lambda$ f ($\lambda$ x (f ((n f) x))))\\
onSucc=($\lambda$ n ($\lambda$ f ($\lambda$ x ((n f) (f x))))\\
(($\lambda$ n ($\lambda$ f ($\lambda$ x (f ((n f) x))))) $\lambda$ f ($\lambda$ x (f x))))\\
$\downarrow \beta$\\
($\lambda$ f ($\lambda$ x (f (($\lambda$ f ($\lambda$ x (f x))) f) x))))\\
$\downarrow \beta$\\
($\lambda$ f ($\lambda$ x (f (($\lambda$ x (f x))) x))))\\
$\downarrow \beta$\\
($\lambda$ f ($\lambda$ x (f (f x)))) $\equiv$ 2!
\end{flushleft}
\end{document}