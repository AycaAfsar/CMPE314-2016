\documentclass{article}
\usepackage[margin=2cm,bottom=2cm]{geometry}
\usepackage{hyperref}
\usepackage{comment}
\usepackage[utf8]{inputenc}
\usepackage{mathtools}
\usepackage{color}
\usepackage{amsmath}
\usepackage{array}
\usepackage{setspace}
\usepackage{soul}
\usepackage{amssymb}
\usepackage{graphicx}

\begin{document}
\title{COMP/CMPE 314 - Principles of Programming Languages - Notes}
\author{Chris Stephenson, Istanbul Bilgi University, Department of Mathematics, and course students}
\date{March,10}
\maketitle

\section*{The "real" word}
\begin{flushleft}
"Evalation \underline{Strategies}"\\
a strange word to use?\\
we make the languages.     
\end{flushleft}

\doublespacing
\begin{flushleft}
When and what and How parameters are evalueted.
\end{flushleft}

\doublespacing
\begin{flushleft}
\underline{When?}\\
In Java: $\rightarrow$
\begin{verbatim}
a,g(b),c+d
myfunc(a,b,c);
myFunc(a++,a++,a++); // order matters
\end{verbatim}
\end{flushleft}

\begin{flushleft}
Are the parameters evaluated left to right or r to l ?\\
In Java left to r\\
In C \underline{undefined}? (Different compilers different answers)
\end{flushleft}

\doublespacing
\begin{flushleft}
\underline{OR} We cannot evaluate the parameters at all. And simply pass the expressions to the function body.\\
(All programming languages do this for \underline{some} things)\\
Racket is a \underline{strict} language, which means parameters are evaluated 
\end{flushleft}

\begin{verbatim}
;#lang lazy
#lang racket
(define (my-if condition t f)
  (cond
    (condition t)
    (else f)))
(define (fac n)
  (my-if (< n 1) 1 (* n (fac (- n 1)))))
\end{verbatim}
%

\begin{flushleft}
\underline{Except} for \underline{if}\\
In Java f(a) \&\& g(b) is a boolean expression\\
They call it "Short circuited evaluation"\\
\color{red}Evaluated left to right and stop when we know the answer
\end{flushleft}

\doublespacing
\begin{flushleft}
\underline{A practical example:}
\end{flushleft}

\begin{verbatim}
int i=0;
	while(i<a.length && a[i]!=0)
	{
	...
	i++;
	}
	while(a[i]!=0 && i<a.length) // after swap, what goes wrong?
\end{verbatim}

\begin{flushleft}
Is evaluation in this language "Strict? Non Strict? Sometimes Strict\\
A different way of binding formal parameters: ..........
\end{flushleft}
\begin{center}
\textbf{\underline{Environments}}
\end{center}

\begin{flushleft}
Substitution is not how the real word works (For reasons of efficency)\\
An environment is an ordered store of identifier / value pairs.\\
For function application we add formal parameter / actual parameter value pairs to the environment and use that environment to evaluate the function body.\\
Our \underline{evaluator} takes the environment is an additional parameters when the evaluator avaluates and identifier, it looks up the value in the environment.
\end{flushleft}

\subsubsection*{Data Definitions}
An environment is empty OR an identifier / value pair plus an environment.
\begin{itemize}
\item look up Enr. environment identifier $\rightarrow$ value or \underline{error}
\item extend Enr. environment identifier $\rightarrow$ environment
\end{itemize}
\begin{verbatim}
(fundef myFunc (a b c) (+ a (* b c))) //function definition
(myFunc 3 7 (+ 2 9)) //function application
\end{verbatim}

\subsubsection*{Evaluate function application}
\begin{verbatim}
(extend Enr(extend Enr(empty Enr, 'a, 3)'b, 7)'c 11)'(+ a (* b c)))
(fundef f1 x(f2 4))
(fundef f2 y (+ x y)) // x is a identifier capture this is called "dynamic scope". Dynamic scope is a bug.

In Java: 
int f2(int y){
	return x+y;
}

evaluate (f1 42) 
- x gets bound to 42
  y gets bound to 4
 (+ x y) looks up these values in the environment
- The answer is 46 wrong!!
\end{verbatim}

\begin{flushleft}
\underline{The solution} (for now)- When we apply a function, do not extend \underline{the current environment} with formal parameter / value bindings. Extend the \underline{empty} environment.
\end{flushleft}

\subsubsection*{Grammer}
\begin{flushleft}

$1)$ $\Lambda\rightarrow\upsilon$

$2)$ $\Lambda\rightarrow(\lambda$ 
$\upsilon$
$\Lambda)$

$3)$ $\Lambda\rightarrow(\Lambda$
$\Lambda)$

Data structure and therefore definitions and programs will reflect this.\\
M[x:=N]\\
\end{flushleft}

\begin{flushleft}
$1)$ M is an identifier\\
a) M=x $\Rightarrow$ M[x	:=N] is N\\
b) M$\neq$x $\Rightarrow$ M[x:=N] is N\\
\end{flushleft}

\begin{flushleft}
$2)$ M is an application just do the two parts\\
\end{flushleft}

\begin{flushleft}
$3)$ M is an function definition\\
Case a is easy $\lambda$yM1, y=x\\
Case b is hard $\lambda$yM1, y$\neq$x\\
\end{flushleft}

\subsubsection*{Church Numerals}
\begin{flushleft}
n $\equiv$ ($\lambda$ f ($\lambda$ x (f (f (f...x))))) //nf's
\end{flushleft}

\begin{flushleft}
ChurchB(n) $\equiv$ (f (ChurchB(n -1)))\\
ChurchB(1) $\equiv$ ($\lambda$ f ($\lambda$ x (f x)))\\
Church(n) $\equiv$ ($\lambda$ f (x x (ChurchB(n))))\\
\end{flushleft}

\begin{flushleft}
Two=($\lambda$ f ($\lambda$ x (f (f x))))\\
Zero=($\lambda$ f ($\lambda$ x x))\\
\end{flushleft}
	
\subsubsection*{A successor function "Succ"}
\begin{flushleft}
Succ=($\lambda$ n ($\lambda$ f ($\lambda$ x (f ((n f) x))))\\
onSucc=($\lambda$ n ($\lambda$ f ($\lambda$ x ((n f) (f x))))
\end{flushleft}

\begin{flushleft}
To add m and n ($\lambda$ m ($\lambda$ n ((m succ) n)))
\end{flushleft}

\end{document}