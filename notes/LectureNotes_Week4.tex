\documentclass{article}
\usepackage[margin=2cm,bottom=2cm]{geometry}
\usepackage{hyperref}
\usepackage{comment}
\usepackage[utf8]{inputenc}
\usepackage{mathtools}
\usepackage{color}
\usepackage{amsmath}
\usepackage{array}
\usepackage{setspace}
\usepackage{soul}
\usepackage{amssymb}
\usepackage{graphicx}

\begin{document}
\title{COMP/CMPE 314 - Principles of Programming Languages - Notes}
\author{Chris Stephenson, Istanbul Bilgi University, Department of Mathematics, and course students}
\date{March,3}
\maketitle

\section*{Lies and Equality (In Java and other languages)}
\begin{flushleft}
"=" does not mean "equals" \\
"\&" does not mean "and"\\
"$||$" does not mean "or"\\   
"!" does not mean "surprise"\\
int does not mean Integer\\
BigInteger does not mean BigInteger\\              
\end{flushleft}

\begin{flushleft}
We use == for "equals"\\
we use \&\& for logical and\\
We use $||$ for logical or\\
\end{flushleft}

\doublespacing
\begin{flushleft}
It means a 32 bit two's complement quality (now most compilers are 64 bit)
\end{flushleft}

\begin{verbatim}
If I declare MyClass a;
a is not necessarily an object
of class MyClass.
\end{verbatim}
%

\subsubsection*{Equality What does == mean ?}
    
\begin{verbatim}
class Equality {
public static void main (String[] args) {
  MyInteger a = new MyInteger(42);
  MyInteger b = new MyInteger(42);
  MyInteger c = new MyInteger(43);
  MyInteger d = c;
  String s1 = "Chris";
  String s2 = "Chris";
  System.out.println("\n a = " + a + ",b = " + b + "\n");
  System.out.println("\n a == b is " + (a == b) + "\n");
  System.out.println("\n s1 == s2 is " + (s1 == s2) + "\n");
  System.out.println("\nd = " + d + "\n");
  System.out.println("\n c == d is " + (c == d) + "\n");
  c.setV(44);
  System.out.println("\n c = " + c +  "\n");
 }
}

class MyInteger {
  int value;
  MyInteger (int v) {
    value = v;
  }
  public String toString () {
    return "" + value;
  }
  public void setV (int v) {
    value = v;
  }
 }
\end{verbatim}    

\doublespacing   
\begin{flushleft}
  It does not mean 'equal in value'(necessarily) 
  \end{flushleft}  
  
  \begin{verbatim}
  String a= "Chris";
  String b= "Chris";
  
  Strings are immutable
  \end{verbatim}
   
    
\begin{flushleft}
Our program
\begin{verbatim}
'((fundef factorial n (if2 n 1(* n (factorial (- n 1))))) (factoaial 3))
\end{verbatim}
\end{flushleft}
$\downarrow$
\begin{verbatim}
parse + desugar
  fdc[name:symbol][arg:symbol][body:s-expression]
  (fdc 'factorial 'n (if(idC n)(numC 1))(numC 1))(multC(numC n)
  				(appC factorial (sub C(idC n)(num 1)
  				
  		(appC 'factorial (numC 3))
\end{verbatim}

\textbf{(if2C \st{(idC 'n)}(numC 1)(mulC \st{(idC 'n)}(appC 'factorial(subC \st{(idC 'n)}(numC 1)))\\
(idc 'n) $\rightarrow$ (numC 3)
}\\


\[ \left( \begin{array}{ccc}
remember & to \\
multiply & by & 3 \\
\end{array} \right)\] 

\[ \left( \begin{array}{ccc}
remember & to \\
multiply & by & 2 \\
\end{array} \right)\]

\[ \left( \begin{array}{ccc}
remember & to \\
multiply & by & 1 \\
\end{array} \right)\] \begin{center}
$\downarrow$
\end{center}

\begin{center}
(mulC \st{(idC 'n)}(appC 'factorial(subC \st{(idC 'n)}(numC 1)))\\
(idC 'n)$\rightarrow$(numC 3)
\end{center}


\subsection*{Transformation}
\begin{flushleft}
($\lambda$ x M) $\equiv$ ($\lambda$ y M [x:=y])\\
$\rightarrow$ [x:=y ]$\rightarrow$ substitude y for x troughout M
\end{flushleft}

\textbf{\begin{flushleft}
$\curvearrowleft$ transformation(optional)
\end{flushleft}}


\begin{flushleft}
($\lambda$ x (f x)) $\overrightarrow{\curvearrowleft}$ f\\
(F y) $\rightarrow$ (f y)\\
(F z) $\rightarrow$ (f z)\\
(F w) $\rightarrow$ (f w)\\
\end{flushleft}

\textbf{\begin{flushleft}
$\beta$ transformation
\end{flushleft}}

\begin{flushleft}
(($\lambda$ x M) y) $\overrightarrow{\beta}$ M [x:=N]\\ 
($\lambda$ x ($\lambda$ y (y x))) $\equiv$ F
\end{flushleft}

\begin{flushleft}
Example \\
((\underline{F M)} N) $\overrightarrow{\beta}$ ((\underline{$\lambda$ y (y M)})N)\\
................ $\overrightarrow{\beta}$ (N M)
\end{flushleft}

\begin{flushleft}
\underline{Some example: } Let M=y N=x \\
((F y) x) $\overrightarrow{\beta}$ (($\lambda$ y (y \underline{y})) x)  // identifier capture has happened (y)\\
................ $\overrightarrow{\beta}$ \underline{(x x)} // disaster !
\end{flushleft}
\end{document}