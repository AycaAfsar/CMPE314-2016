
\documentclass{article}
\usepackage[margin=2cm,bottom=2cm]{geometry}
\usepackage{hyperref}
\usepackage{comment}
\usepackage[utf8]{inputenc}
\usepackage{graphicx}
\usepackage{mathtools}
\usepackage[normalem]{ulem}
\usepackage{setspace}
\usepackage{MnSymbol}
\usepackage{soul}

\newcommand\tab[1][1cm]{\hspace*{#1}}
\DeclareMathSizes{10}{10}{10}{10}

\begin{document}
\title{COMP/CMPE 314 - Principles of Programming Languages - QUIZE SOLUTIONS}
\author{Chris Stephenson, Istanbul Bilgi University, Department of Mathematics, and course students}
\maketitle
\textbf{QUIZ 3}\\ 
\textbf{Question 1}\\
$\Lambda$ $\rightarrow$ V\\
$\Lambda$ $\rightarrow$($\lambda$ V $\Lambda$)\\
$\Lambda$ $\rightarrow$($\Lambda$ $\Lambda$)\\
V$\rightarrow$a\\
V$\rightarrow$b\\
V$\rightarrow$VV\\
\textbf{Solution 1}\\
((z b)a)$\rightarrow$ valid\\
($\lambda$ (b) z)$\rightarrow$ invalid\\
($\lambda$ a b)$\rightarrow$ valid\\
(b($\lambda$ z b))$\rightarrow$ valid\\
(($\lambda$ z b) z)$\rightarrow$ valid\\
((z b a) a)$\rightarrow$ invalid\\
ab$\rightarrow$ valid\\
(b)$\rightarrow$ invalid\\
((z a))$\rightarrow$ invalid\\
(a b z)$\rightarrow$ invalid\\
(a($\lambda$ a ($\lambda$ a a)))$\rightarrow$ valid\\
z $\rightarrow$valid\\
b z$\rightarrow$ invalid\\
(a b)$\rightarrow$ valid\\
(($\lambda$ z (z z) z)$\rightarrow$ valid\\
(($\lambda$ a (a (b z a))) z)$\rightarrow$ invalid\\
\textbf{Question 2}\\
Write a set of tests for the procedure performing substitıtion in your project 03.\\
\textbf{Solution 2} \\
(test (subst (numC 5) 'x (plusC (idC 'x) (numC 3))) (plusC (numC 5) (numC 3)))\\
(test (subst (numC 6) 'x (multC (idC 'x) (idC 'x))) (multC (numC 6) (numC 6)))\\
\textbf{Question 3}\\
In chapter 5 of the text book, what are the steps in evaluating a function application\\
\textbf{Solution 3}\\
(funapp (id=) (exp))\\
1)Look up the function name in the store of function definitions.\\
2)Evaluate the actual paramater.\\
3)Substituon the actual paramater for the formal parameter in body of definition.\\
4)Evaluate body.\\
\maketitle
\textbf{QUIZ 5}\\
\textbf{Question 1}\\
 $\Lambda$$\rightarrow$V\\
 $\Lambda$$\rightarrow$($\lambda$ V $\Lambda$)\\
 $\Lambda$$\rightarrow$( $\Lambda$ $\Lambda$)\\
V$\rightarrow$a\\
V$\rightarrow$b\\
V$\rightarrow$z\\
V$\rightarrow$VV\\
\textbf{Solution 1}\\
((z b) a)$\rightarrow$ valid\\
((b z))$\rightarrow$ invalid\\
(z b a)$\rightarrow$ invalid\\
($\lambda$ (a) a)$\rightarrow$ invalid\\
($\lambda$ z b)$\rightarrow$ valid\\
(a ($\lambda$ b z))$\rightarrow$ valid\\
ab$\rightarrow$ valid\\
b $\rightarrow$valid\\
z a $\rightarrow$invalid\\
(z b)$\rightarrow$ valid\\
(z) $\rightarrow$invalid\\
(($\lambda$ b a) z)$\rightarrow$ valid\\
((a b a) z)$\rightarrow$ invalid\\
(a a($\lambda$ z ($\lambda$ z z)))$\rightarrow$ valid\\
(($\lambda$ b (a b)) z)$\rightarrow$ valid\\
(($\lambda$ b (a (b))) z)$\rightarrow$ invalid\\
\textbf{Question 2}\\
Calculate the set of Free Identifiers, which may be empty, for each of the $\lambda$-calculus sentences below.\\
Write the set in curly brackets, so {a, b} represents the set containing a and b.\\
\textbf{Solution 2}\\
($\lambda$ x ($\lambda$ y ($\lambda$ z(($\lambda$ x (y y))($\lambda$ y (x x))))))$\rightarrow$ \{\} \\
(($\lambda$ b b) b)$\rightarrow$ \{b\} \\
a $\rightarrow$ \{a\} \\
($\lambda$ x ($\lambda$ y (($\lambda$ x (y y))($\lambda$ y (x x)))))$\rightarrow$ \{\} \\
(($\lambda$ b c) b)$\rightarrow$\{c b\} \\
($\lambda$ x (y y))($\lambda$ y (x x)))$\rightarrow$\{x y\} \\
($\lambda$ y($\lambda$ x(y y)))$\rightarrow$\{\} \\
($\lambda$ b ((a b)(b c)))$\rightarrow$\{a c\} \\
($\lambda$ x (x x))($\lambda$ x(x x)))$\rightarrow$\{\} \\
((a b)(b c))$\rightarrow$\{a b c\} \\
(($\lambda$ x(x x))($\lambda$ y (y y)))$\rightarrow$\{\} \\
(a b)$\rightarrow$ \{a b\} \\
($\lambda$ a a)$\rightarrow$ \{\} \\
($\lambda$ x ($\lambda$ y (y y)))$\rightarrow$\{\} \\
($\lambda$ a b)$\rightarrow$ \{b\} \\
\textbf{Question 3}\\
Write a tail recursive factorial program that calculates the factorial of 7 using your target language from project04.\\
\textbf{Solution 3}\\
(define (fact p acc)\\
  \tab(cond\\
   \tab\tab $[$ (=p 1) acc ] \\
    \tab\tab$[$else (fact (-p 1)(*acc p))]\\
))\\
\maketitle
\textbf{QUIZ 7}\\ 
\textbf{Question 1}\\
Calculate the set of Free Identifiers, which may be empty, for each of the lambda-calculus sentences below.\\
Write the set in curly brackets, so {a, b} represents the set containing a and b\\
\textbf{Solution 1}\\
(($\lambda$ x(x x))($\lambda$ y (y y)))$\rightarrow$ \{\} \\
(a b)$\rightarrow$ \{a b\} \\
($\lambda$ a a)$\rightarrow$ \{\} \\
($\lambda$ x ($\lambda$ y (y y)))$\rightarrow$ \{\} \\
($\lambda$ a b)$\rightarrow$ \{b\} \\
($\lambda$ x ($\lambda$ y ($\lambda$ z(($\lambda$ x (y y))($\lambda$ y (x x))))))$\rightarrow$\{\} \\
(($\lambda$ b b) b)$\rightarrow$ \{b\}\\
a $\rightarrow$ \{a\}
($\lambda$ x ($\lambda$ y (($\lambda$ x (y y))($\lambda$ y (x x)))))$\rightarrow$ \{\} \\ 
(($\lambda$ b c) b)$\rightarrow$ \{c b\} \\
($\lambda$ x (y y))($\lambda$ y (x x)))$\rightarrow$ \{x y\} \\
($\lambda$ y($\lambda$ x(y y)))$\rightarrow$\{\} \\
($\lambda$ b ((a b)(b c)))$\rightarrow$ \{a c\} \\
($\lambda$ x (x x))($\lambda$ x(x x)))$\rightarrow$ \{\} \\
((a b)(b c))$\rightarrow$ \{a b c\} \\
\textbf{Question 2}\\
Calculate the results ıf a beta-transformation on each of the lambda-sentences below.\\ 
You only need to perform one transformtion even if a further transformation becomes possible.\\
\textbf{Solution 2}\\
(($\lambda$ x($\lambda$ y(y x))) y)$\rightarrow$ ($\lambda$ y (y y))\\
(($\lambda$ x($\lambda$ y(y x))) z)$\rightarrow$ ($\lambda$ y (y z))\\
(($\lambda$ x (x x))($\lambda$ y (y y))) $\rightarrow$ ($\lambda$ y (y y))($\lambda$ y (y y)))\\
(($\lambda$ a (b b)) c)$\rightarrow$ (b b)\\
(($\lambda$ z (z z) p))(q($\lambda$ x x)))$\rightarrow$ (((q ($\lambda$ x x))(q ($\lambda$ x x))) p)\\



   

\end{document}
